\documentclass[class=article, crop=false]{standalone}

\usepackage{style}
\usepackage{standalone}

\begin{document}

\section*{Introduction}
The $V$ programming language is a functional language with eager left-to-right evaluation.
It has a simple I/O system supporting only direct string operations.
It is a trait based strongly and statically typed language supporting both explicit and implicit typing.


This document both specifies the $V$ language and shows its implementation in F\#.
It is divided into 6 categories:
\begin{enumerate}
    \item Abstract Syntax and Semantics

        This defines the abstract syntax and semantics of the language.
        It only contains the bare minimum for the language to function, without any syntactic sugar.

    \item Extended Language

        Defines the extended abstract syntax tree and its translation into the core language.

    \item Concrete Syntax

        Defines the concrete syntax for the language, describing what are valid expressions and programs.

    \item Language Guide

        A guide for programming in $V$.
        This defines all operators, syntactic sugar and other aspects of the language, along with short user-friendly explanations of each language feature (and limitation).

    \item Standard Library

        Describes all functions provided in the $V$ standard library.

    \item Changelog

         Lists the changes done to the language in each version.
\end{enumerate}

\end{document}
